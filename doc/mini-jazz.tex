\documentclass[a4paper]{article}
\usepackage[utf8]{inputenc}
\usepackage[french]{babel}
\usepackage[a4paper]{geometry}
\usepackage{hyperref}
\usepackage{listings}
\usepackage{amsthm}
\usepackage[a4paper]{geometry}

\usepackage{mj_lang}

% fontes tt avec gras (mots-clés)
\renewcommand{\ttdefault}{txtt}

\newtheorem{question}{Question}

%\title{TP1: Un interprète de net-list}
\title{\minijazz{} and net-list}
%\author{Cédric Pasteur \texttt{<cedric.pasteur@ens.fr>}}
\author{Timothy Bourke \texttt{<Timothy.Bourke@ens.fr>}}

\begin{document}

\maketitle

\section{Le langage \minijazz}

\subsection{Description du langage}

Le langage \minijazz{} est un langage de description de circuits digitaux synchrones. Il permet de décrire des circuits de façon récursive et hiérarchique et d'utiliser des nappes de fils. Voici par exemple la description d'un \emph{full-adder}:
\begin{lstlisting}
fulladder(a,b,c) = (s, r) where
  s = (a ^ b) ^ c;
  r = (a & b) + ((a ^ b) & c);
end where
\end{lstlisting}

\subsubsection*{Bloc}

Un circuit est décrit par un ensemble de blocs. Un bloc est défini par son nom \texttt{f}, ses entrées \lstinline{a_1, .., a_p}, ses sorties \lstinline{o_1, ..., o_q} et un ensemble \texttt{D} d'équations définissant les sorties en fonction des entrées. Il est possible de définir des variables locales sans les déclarer.
\begin{lstlisting}
f(a_1, ..., a_p) = (o_1, ..., o_q) where
  D
end where
\end{lstlisting}

Les équations sont soit de la forme \lstinline+x = e+ où e est une expression, soit l'instantiation d'un autre bloc de la forme \lstinline+(y_1, ..., y_q) =  g(x_1, .., x_p)+, soit un bloc conditionnel \lstinline+if c then D1 else D2 end if+ où \lstinline+c+ est une expression statique booléenne.

Les opérateurs prédefinis sont l'inverseur (\lstinline+not+), le \emph{et} 
logique (\lstinline+&+ ou \lstinline+and+), le \emph{ou} logique 
(\lstinline+or+ ou~\lstinline{+}), le \emph{ou exclusif} (\lstinline+xor+ ou 
\lstinline+^+), \lstinline+nand+ et le multiplexeur 1 bit (\lstinline+mux+). 
Les constantes \lstinline+true+ (ou \texttt{1}) et \lstinline+false+ (ou 
\texttt{0}) sont également disponibles.

\subsubsection*{Nappes de fils}

On peut déclarer une variable \lstinline+x+ (entrée ou sortie) comme étant 
une nappe de \lstinline+n+ fils avec la syntaxe \lstinline+x : [n]+. On peut 
alors accéder au ième fil avec \lstinline+x[i]+ (les indices commencent à 0) 
ou extraire une sous nappe avec \lstinline+x[i1 .. i2]+. On peut aussi 
écrire \lstinline+x[i1..]+ (resp. \lstinline+x[..i2]+) comme raccourci de 
\lstinline+x[i1..n-1]+ (resp. \lstinline+x[0..i2]+) si le tableau est de 
taille \texttt{n}. Il est également utile de concaténer des nappes avec 
l'opérateur \lstinline+.+. 

Voici par exemple un bloc qui décale sa sortie de 1 bit vers la gauche:
\begin{lstlisting}
shift_left(a:[4]) = (o:[4]) where
  o = a[1..3] . 0;
end where
\end{lstlisting}


\subsubsection*{Paramétricité}
\label{ref:static}

La taille d'un tableau est donné par une \emph{expression statique}, qui est soit une constant globale, soit un paramètre d'un bloc, soit un opérateur statique (\lstinline{+, -, *, /, ^, =, <=}) appliqué à deux expression statiques. Les constantes globales sont déclarées par:
\begin{lstlisting}
const n = 32
\end{lstlisting}

Les paramètres statiques d'un bloc sont donnés entre crochets après son nom. Combinés avec les blocs conditionnels et la récursion, ils permettent de créer des circuits décrits de façon récursive. Un autre élément utile est la valeur \lstinline+[]+, qui est une nappe de taille 0. On peut par exemple définir un additionneur $n$ bits à partir d'un \texttt{fulladder} et d'un additionneur $n-1$ bits:
\begin{lstlisting}
adder<n>(a:[n], b:[n], c_in) = (o:[n], c_out) where
  if n = 0 then
    o = [];
    c_out = 0
  else
    (s_n1, c_n1) = adder<n-1>(a[1..], b[1..], c_in);
    (s_n, c_out) = fulladder(a[0], b[0], c_n1);
    o = s_n . s_n1
  end if
end where
\end{lstlisting}


\subsubsection*{Primitives mémoire}

Il est possible de déclarer des blocs mémoire, ROM ou RAM, avec la syntaxe suivante:
\begin{lstlisting}
o1 = ram<addr_size, word_size>(read_addr, write_enable, write_addr, write_data);
o2 = rom<addr_size, word_size>(read_addr);
\end{lstlisting}
\lstinline+word_size+ est la taille des mots mémoires, c'est-à-dire le 
nombre de bits lus à chaque cycle, qui est aussi la taille de la sortie. 
\lstinline+addr_size+ est la taille des adresses en mémoire, c.-à-d. la 
mémoire contient $2^{\mathtt{addr\_size}}$ mots. Une ROM prend en entrée une 
adresse et renvoie le mot stocké à cette adresse. Une RAM fait le même chose 
mais prend aussi en entrée un bit indiquant s'il faut écrire 
(\texttt{write\_enable}), suivi de l'adresse et du mot à écrire. La RAM 
fonctionne comme un registre: on peut lire et écrire au même emplacement de 
la mémoire (c.-à-d. \lstinline+read_addr = write_addr+ et 
\lstinline+write_enable = 1+) sans qu'il y ait de boucle combinatoire.  La 
valeur lue sera celle dans la mémoire à l'instant précédent.


\subsection{Compilateur}

%\subsubsection*{Compilation}
%
%Pour obtenir l'exécutable du compilateur, lancer dans le dossier \texttt{minijazz/}:
%\begin{verbatim}
%> ocamlbuild mjc.byte
%\end{verbatim}
%
%
%\subsubsection*{Exécution}

Le compilateur \minijazz{} génère à partir d'un programme une net-list non ordonnée, ne contenant plus de blocs et dont chaque équation ne contient plus qu'une seule opération. Il faut spécifier le bloc principal du circuit avec l'option \texttt{-m} (par défaut, il s'appelle \texttt{main}).
\begin{verbatim}
> ./mjc.byte -m mon_bloc mon_fichier.mj
\end{verbatim}
Le fichier généré est un simple fichier texte décrivant une \emph{net-list}, avec l'extension \texttt{.net}, dont le contenu est décrit dans la partie suivante. 

\section{Le langage de \emph{net-list}}

\subsection{Syntaxe concrète}

Une \emph{net-list} a la forme suivante:
\begin{lstlisting}[language=nl]
INPUT inputs
OUTPUT outputs
VAR vars IN
  D
\end{lstlisting}
\texttt{inputs} et \texttt{outputs} sont les entrées et sorties du circuit, sous forme d'une liste d'identifiants. \texttt{vars} donne le type des entrées, sorties et variables locales (avec la même syntaxe que pour \minijazz{}). Enfin, \emph{D} représente la liste des portes du circuit. Chaque équation ne contient plus qu'une seule opération élémentaire, soit une opération booléenne (\texttt{OR}, \texttt{XOR}, \texttt{AND}, \texttt{NAND}), soit définir un registre (\texttt{REG}) ou une mémoire (\texttt{RAM}, \texttt{ROM}), soit une opération sur des nappes de fils (\texttt{SELECT}, \texttt{SLICE}, \texttt{CONCAT}). Voici par exemple la \emph{net-list} générée par le compilateur pour le circuit \texttt{Fulladder}:
\begin{lstlisting}
INPUT a, b, c
OUTPUT s, r
VAR
  _l_1, _l_3, _l_4, _l_5, a, b, c, r, s
IN
r = OR _l_3 _l_5
s = XOR _l_1 c
_l_1 = XOR a b
_l_3 = AND a b
_l_4 = XOR a b
_l_5 = AND _l_4 c
\end{lstlisting}

\subsection{Syntaxe abstraite}

La syntaxe abstraite du langage de \emph{net-list}, c'est-à-dire l'ensemble 
des types représentant un circuit, est donné dans le fichier 
\texttt{netlist\_ast.ml}. Un analyseur lexical (\texttt{netlist\_lexer.mll}) 
et un analyseur syntaxique (\texttt{netlist\_parser.mly}) permettent de 
traduire un fichier texte dans la syntaxe concrète en la structure de 
données correspondante. On utilisera pour cela directement la fonction 
suivante définie dans le fichier \texttt{netlist.ml}, qui attend un nom de 
fichier et renvoie le circuit correspondant:
\begin{lstlisting}[language={[objective]caml}]
val read_file : string -> Netlist_ast.program
\end{lstlisting}

Pour faire l'opération inverse, on utilisera la fonction suivante, définie dans \texttt{netlist\_printer.ml}:
\begin{lstlisting}[language={[objective]caml}, xrightmargin=0.5cm]
val print_program : out_channel -> Netlist_ast.program -> unit
\end{lstlisting}

\end{document}
\clearpage{}
\section{TP}

Les fichiers nécessaires au TP peuvent être récupérés à l'adresse: 
\url{http://www.di.ens.fr/~bourke/}. Vous devrez utiliser le compilateur 
\minijazz{} (\texttt{mjc.byte}) et un simulateur de net-list 
(\texttt{netlist\_simulator.byte}), qui sont contenus dans l'archive. Les 
sources du compilateur sont également disponibles au meme endroit si vous 
désirez le modifier pour la suite de votre projet.\footnote{Mais pas celles 
du simulateur, qui constitue la première partie du projet}

\subsection{Ordonnancement}

Le compilateur \minijazz{} génère des net-list dans un ordre quelconque. La première étape pour les simuler est d'ordonner les équations afin de pouvoir les exécuter séquentiellement. Cela se fera par un tri topologique du graphe de dépendences du circuit.


\begin{question}
Le fichier \texttt{graph.ml} contient une implémentation simple de graphes orientés impératifs. Complétez le fichier pour ajouter la détection de cycles et le tri topologique.
\begin{lstlisting}[language={[objective]caml}]
(* Detection de cycles *)
val has_cycle : 'a graph -> bool 
(* Renvoie les noeuds dans un ordre topologique *)
val topological : 'a graph -> 'a list
\end{lstlisting}
\end{question}

Pour tester votre module, compilez le programme de test avec:
\begin{verbatim}
> ocamlbuild graph_test.byte
\end{verbatim}
et vérifiez que le résultat est bien celui attendu.\\

On va maintenant construire le graphe de dépendences d'un circuit. Les noeuds de ce graphe sont les variables du circuit et il existe un chemin de \texttt{x} vers \texttt{y} si on a besoin de la valeur de \texttt{y} pour calculer \texttt{x}.
\begin{question}
Ecrivez une fonction \texttt{read\_exp} (dans le fichier \texttt{scheduler.ml}) qui renvoie la liste des variables lues par une équation, de signature:
\begin{lstlisting}[language={[objective]caml}, xrightmargin=0.5cm]
val read_exp : Netlist_ast.equation -> Netlist_ast.ident list 
\end{lstlisting}
(On ignorera pour l'instant le cas des registres et des mémoires qui seront traités dans la question~\ref{q:sch_reg})
\end{question}

\begin{question}
Ecrivez une fonction \texttt{schedule} (dans \texttt{scheduler.ml}) qui prend en entrée une net-list et renvoie la net-list ordonnée afin de permettre une exécution séquentielle:
\begin{lstlisting}[language={[objective]caml}]
val schedule : Netlist_ast.program -> Netlist_ast.program
\end{lstlisting}
Votre fonction devra lever l'exception \lstinline+Combinational_cycle+ si jamais la net-list contient une boucle combinatoire (On ignorera pour l'instant le cas des registres et des mémoires qui seront traités dans la question~\ref{q:sch_reg}).
\end{question}

Pour tester votre ordonnanceur, compilez le programme avec:
\begin{verbatim}
> ocamlbuild scheduler_test.byte
\end{verbatim}
puis exécutez les programmes de test. Par exemple:
\begin{verbatim}
> ./scheduler_test.byte test/fulladder.net
\end{verbatim}
L'exécution du programme génère une nouvelle \emph{net-list} \texttt{fulladder\_sch.net}, puis exécute le simulateur sur cette net-list (sauf si l'option \texttt{-print} est donnée). On peut choisir le nombre de pas effectués par le simulateur avec l'option \texttt{-n <nombre>}. \\

Les registres permettent de casser les cycles de causalité, puisque l'on peut lire la valeur d'un registre avant d'avoir écrit la nouvelle valeur.
\begin{question}
\label{q:sch_reg}
Ajoutez la prise en compte des registres et des mémoires à votre ordonnanceur.
\end{question}

\subsection{Un interprète de net-list}

Un \emph{interprète} est un programme qui évalue une net-list en utilisant un \emph{environnement} pour stocker les valeurs calculées précédemment. Cet environnment associe à un identifiant (\texttt{Netlist\_ast.ident}) une valeur (\texttt{Netlist\_ast.value}).

\begin{question}
Quelle est la structure générale de l'interprète ne gérant que les circuits purement combinatoires ? Ecrivez le pseudo-code correspondant.
\end{question}

\begin{question}
Comment gérer les registres et les mémoires ?
\end{question}


%\subsection{Un interprète simple}
%
%Le fichier \texttt{netlist\_step.ml} contient le squelette d'un interprète basique. Il utilise un environnent \texttt{env}, c'est-à-dire une \texttt{Map.t} associant à un identifiant (\texttt{Netlist\_ast.ident}) une valeur (\texttt{Netlist\_ast.value}), pour stocker les valeurs de variables calculées précédemment. 
%
%\begin{question}
%Compléter les cas manquants dans la fonction d'évaluation des expressions \texttt{eval}.
%\end{question}
%Vous pouvez tester votre évaluateur en lan\c{}ant \verb+./netlist_simulator.byte+ avec l'option \texttt{-i} pour utiliser l'interprète au lieu de générer un fichier.
%
%\begin{question}
%Ajouter le gestion des registres à l'interprète. 
%\end{question}

\end{document}
